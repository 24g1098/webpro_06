\documentclass[uplatex,dvipdfmx]{jsarticle}

\usepackage[uplatex,deluxe]{otf} % UTF
\usepackage[noalphabet]{pxchfon} % must be after otf package
\usepackage{stix2} %欧文&数式フォント
\usepackage[fleqn,tbtags]{mathtools} % 数式関連 (w/ amsmath)
\usepackage{hira-stix} % ヒラギノフォント&STIX2 フォント代替定義(Warning回避)

\usepackage{listings} % listingsパッケージを読み込む

\lstset{
  basicstyle={\ttfamily}, % 基本となるフォントスタイル(typewriterフォント)
  identifierstyle={\small}, % 識別子のフォントスタイル
  commentstyle={\small\itshape}, % コメントのフォントスタイル
  keywordstyle={\small\bfseries}, % キーワード(予約語)のフォントスタイル
  ndkeywordstyle={\small}, % 追加のキーワードのフォントスタイル
  stringstyle={\small\ttfamily}, % 文字列のフォントスタイル
  frame={tb}, % ソースコードを囲む枠のスタイル(上下に線を表示)
  breaklines=true, % 長い行を改行するかどうか
  columns=[l]{fullflexible}, % ソースコードの文字幅の調整方法
  numbers=left, % 行番号を左側に表示する
  xrightmargin=0zw, % 右側の余白
  xleftmargin=3zw, % 左側の余白
  numberstyle={\scriptsize}, % 行番号のフォントサイズ
  stepnumber=1, % 行番号を表示するステップ
  numbersep=1zw, % 行番号とソースコード間のスペース
  lineskip=-0.5ex % 行間の調整
}

\begin{document}

\title{メモりある 仕様書} %システム名 仕様書 という形式にする
\author{24G1098 手代木巧}
\date{2025年1月7日}
\maketitle

\section{概要}
本仕様書では、シングルページアプリケーション(SPA)「メモりある」の機能および実装について説明します。本アプリケーションは、メモの作成・閲覧・削除が可能な簡易的なメモ管理システムです。

\section{利用者向け}
\subsection{アプリケーション概要}
本アプリケーションは、利用者が簡単にメモを作成し、保存・削除することができるシステムです。

\subsection{主な機能}
\begin{itemize}
    \item メモの作成:入力欄に文字を入力し「メモを追加」ボタンを押すとメモが保存されます。
    \item メモの一覧表示:作成したメモがリスト形式で画面下部に表示されます。
    \item メモの削除:各メモの右側にある「メモを消去」ボタンでメモを削除できます。
\end{itemize}

\subsection{使用方法}
\begin{enumerate}
    \item アプリケーションをブラウザで開きます。
    \item 入力欄にメモを記入し、「メモを追加」ボタンを押して保存します。
    \item 画面下部のメモ一覧から削除したいメモを「メモを消去」ボタンで削除します。
\end{enumerate}

\section{管理者向け}
\subsection{管理業務}
\begin{itemize}
    \item サーバーの起動:Node.jsでサーバーを実行します。
    \item データ管理:アプリケーションのメモデータはサーバーで一時的に保持されます。
    \item エラー対応:エラー発生時はログを確認して対応します。
\end{itemize}

\subsection{実行方法}
\begin{lstlisting}[caption=サーバーの起動コマンド]
node sav.js
\end{lstlisting}

\subsection{メンテナンス手順}
\begin{itemize}
    \item サーバーの再起動:
    \begin{lstlisting}
ctrl + c # サーバー停止
node sav.js # サーバー再起動
    \end{lstlisting}
    \item 静的ファイルの変更時は、更新内容が反映されているか確認してください。
\end{itemize}

\section{開発者向け}
\subsection{通信仕様}
\begin{itemize}
    \item メモの追加(POST \texttt{/memos})
    \begin{lstlisting}[caption=リクエスト例]
{
  "text": "新しいメモの内容"
}
    \end{lstlisting}

    \begin{lstlisting}[caption=レスポンス例]
{
  "success": true,
  "memos": ["新しいメモの内容"]
}
    \end{lstlisting}

    \item メモの削除(POST \texttt{/memos/delete})
    \begin{lstlisting}[caption=リクエスト例]
{
  "index": 0
}
    \end{lstlisting}

    \begin{lstlisting}[caption=レスポンス例]
{
  "success": true,
  "memos": []
}
    \end{lstlisting}
\end{itemize}

\subsection{開発環境}
\begin{itemize}
    \item 使用技術:
    \begin{itemize}
        \item Node.js(Express)
        \item HTML/CSS/JavaScript
    \end{itemize}
    \item 環境構築:
    \begin{lstlisting}[caption=環境構築コマンド]
npm install # 必要な依存関係をインストール
node sav.js # サーバーを起動
    \end{lstlisting}
\end{itemize}

\end{document}
